\batchmode
\documentclass[onecolumn]{article}
\RequirePackage{ifthen}

 
\usepackage[utf8]{inputenc} 
\usepackage{abstract} 
\usepackage{graphicx} 
\usepackage[font={footnotesize}]{caption}
\usepackage{wrapfig,framed}
\usepackage[margin=1.55cm]{geometry} 
\usepackage{float}
\usepackage{hyperref}
\usepackage[hyphenbreaks]{breakurl}
\makeatletter

%
\renewcommand{\abstractname}{} %
%
\renewcommand{\absnamepos}{empty} %
\title{\Huge r0oT \\
\vspace{0.4 cm} \small a network to achieve global coordination\\and true civilization. 
(Manifestum v2.3)} 


\usepackage[dvips]{color}


\pagecolor[gray]{.7}

\usepackage[]{inputenc}



\makeatletter
\AtBeginDocument{\makeatletter
\input /home/aleix/world/spirit/r0oT/manifestum_2/r0oTv2.3.aux
\makeatother
}

\makeatletter
\count@=\the\catcode`\_ \catcode`\_=8 
\newenvironment{tex2html_wrap}{}{}%
\catcode`\<=12\catcode`\_=\count@
\newcommand{\providedcommand}[1]{\expandafter\providecommand\csname #1\endcsname}%
\newcommand{\renewedcommand}[1]{\expandafter\providecommand\csname #1\endcsname{}%
  \expandafter\renewcommand\csname #1\endcsname}%
\newcommand{\newedenvironment}[1]{\newenvironment{#1}{}{}\renewenvironment{#1}}%
\let\newedcommand\renewedcommand
\let\renewedenvironment\newedenvironment
\makeatother
\let\mathon=$
\let\mathoff=$
\ifx\AtBeginDocument\undefined \newcommand{\AtBeginDocument}[1]{}\fi
\newbox\sizebox
\setlength{\hoffset}{0pt}\setlength{\voffset}{0pt}
\addtolength{\textheight}{\footskip}\setlength{\footskip}{0pt}
\addtolength{\textheight}{\topmargin}\setlength{\topmargin}{0pt}
\addtolength{\textheight}{\headheight}\setlength{\headheight}{0pt}
\addtolength{\textheight}{\headsep}\setlength{\headsep}{0pt}
\setlength{\textwidth}{349pt}
\newwrite\lthtmlwrite
\makeatletter
\let\realnormalsize=\normalsize
\global\topskip=2sp
\def\preveqno{}\let\real@float=\@float \let\realend@float=\end@float
\def\@float{\let\@savefreelist\@freelist\real@float}
\def\liih@math{\ifmmode$\else\bad@math\fi}
\def\end@float{\realend@float\global\let\@freelist\@savefreelist}
\let\real@dbflt=\@dbflt \let\end@dblfloat=\end@float
\let\@largefloatcheck=\relax
\let\if@boxedmulticols=\iftrue
\def\@dbflt{\let\@savefreelist\@freelist\real@dbflt}
\def\adjustnormalsize{\def\normalsize{\mathsurround=0pt \realnormalsize
 \parindent=0pt\abovedisplayskip=0pt\belowdisplayskip=0pt}%
 \def\phantompar{\csname par\endcsname}\normalsize}%
\def\lthtmltypeout#1{{\let\protect\string \immediate\write\lthtmlwrite{#1}}}%
\newcommand\lthtmlhboxmathA{\adjustnormalsize\setbox\sizebox=\hbox\bgroup\kern.05em }%
\newcommand\lthtmlhboxmathB{\adjustnormalsize\setbox\sizebox=\hbox to\hsize\bgroup\hfill }%
\newcommand\lthtmlvboxmathA{\adjustnormalsize\setbox\sizebox=\vbox\bgroup %
 \let\ifinner=\iffalse \let\)\liih@math }%
\newcommand\lthtmlboxmathZ{\@next\next\@currlist{}{\def\next{\voidb@x}}%
 \expandafter\box\next\egroup}%
\newcommand\lthtmlmathtype[1]{\gdef\lthtmlmathenv{#1}}%
\newcommand\lthtmllogmath{\dimen0\ht\sizebox \advance\dimen0\dp\sizebox
  \ifdim\dimen0>.95\vsize
   \lthtmltypeout{%
*** image for \lthtmlmathenv\space is too tall at \the\dimen0, reducing to .95 vsize ***}%
   \ht\sizebox.95\vsize \dp\sizebox\z@ \fi
  \lthtmltypeout{l2hSize %
:\lthtmlmathenv:\the\ht\sizebox::\the\dp\sizebox::\the\wd\sizebox.\preveqno}}%
\newcommand\lthtmlfigureA[1]{\let\@savefreelist\@freelist
       \lthtmlmathtype{#1}\lthtmlvboxmathA}%
\newcommand\lthtmlpictureA{\bgroup\catcode`\_=8 \lthtmlpictureB}%
\newcommand\lthtmlpictureB[1]{\lthtmlmathtype{#1}\egroup
       \let\@savefreelist\@freelist \lthtmlhboxmathB}%
\newcommand\lthtmlpictureZ[1]{\hfill\lthtmlfigureZ}%
\newcommand\lthtmlfigureZ{\lthtmlboxmathZ\lthtmllogmath\copy\sizebox
       \global\let\@freelist\@savefreelist}%
\newcommand\lthtmldisplayA{\bgroup\catcode`\_=8 \lthtmldisplayAi}%
\newcommand\lthtmldisplayAi[1]{\lthtmlmathtype{#1}\egroup\lthtmlvboxmathA}%
\newcommand\lthtmldisplayB[1]{\edef\preveqno{(\theequation)}%
  \lthtmldisplayA{#1}\let\@eqnnum\relax}%
\newcommand\lthtmldisplayZ{\lthtmlboxmathZ\lthtmllogmath\lthtmlsetmath}%
\newcommand\lthtmlinlinemathA{\bgroup\catcode`\_=8 \lthtmlinlinemathB}
\newcommand\lthtmlinlinemathB[1]{\lthtmlmathtype{#1}\egroup\lthtmlhboxmathA
  \vrule height1.5ex width0pt }%
\newcommand\lthtmlinlineA{\bgroup\catcode`\_=8 \lthtmlinlineB}%
\newcommand\lthtmlinlineB[1]{\lthtmlmathtype{#1}\egroup\lthtmlhboxmathA}%
\newcommand\lthtmlinlineZ{\egroup\expandafter\ifdim\dp\sizebox>0pt %
  \expandafter\centerinlinemath\fi\lthtmllogmath\lthtmlsetinline}
\newcommand\lthtmlinlinemathZ{\egroup\expandafter\ifdim\dp\sizebox>0pt %
  \expandafter\centerinlinemath\fi\lthtmllogmath\lthtmlsetmath}
\newcommand\lthtmlindisplaymathZ{\egroup %
  \centerinlinemath\lthtmllogmath\lthtmlsetmath}
\def\lthtmlsetinline{\hbox{\vrule width.1em \vtop{\vbox{%
  \kern.1em\copy\sizebox}\ifdim\dp\sizebox>0pt\kern.1em\else\kern.3pt\fi
  \ifdim\hsize>\wd\sizebox \hrule depth1pt\fi}}}
\def\lthtmlsetmath{\hbox{\vrule width.1em\kern-.05em\vtop{\vbox{%
  \kern.1em\kern0.8 pt\hbox{\hglue.17em\copy\sizebox\hglue0.8 pt}}\kern.3pt%
  \ifdim\dp\sizebox>0pt\kern.1em\fi \kern0.8 pt%
  \ifdim\hsize>\wd\sizebox \hrule depth1pt\fi}}}
\def\centerinlinemath{%
  \dimen1=\ifdim\ht\sizebox<\dp\sizebox \dp\sizebox\else\ht\sizebox\fi
  \advance\dimen1by.5pt \vrule width0pt height\dimen1 depth\dimen1 
 \dp\sizebox=\dimen1\ht\sizebox=\dimen1\relax}

\def\lthtmlcheckvsize{\ifdim\ht\sizebox<\vsize 
  \ifdim\wd\sizebox<\hsize\expandafter\hfill\fi \expandafter\vfill
  \else\expandafter\vss\fi}%
\providecommand{\selectlanguage}[1]{}%
\makeatletter \tracingstats = 1 


\begin{document}
\pagestyle{empty}\thispagestyle{empty}\lthtmltypeout{}%
\lthtmltypeout{latex2htmlLength hsize=\the\hsize}\lthtmltypeout{}%
\lthtmltypeout{latex2htmlLength vsize=\the\vsize}\lthtmltypeout{}%
\lthtmltypeout{latex2htmlLength hoffset=\the\hoffset}\lthtmltypeout{}%
\lthtmltypeout{latex2htmlLength voffset=\the\voffset}\lthtmltypeout{}%
\lthtmltypeout{latex2htmlLength topmargin=\the\topmargin}\lthtmltypeout{}%
\lthtmltypeout{latex2htmlLength topskip=\the\topskip}\lthtmltypeout{}%
\lthtmltypeout{latex2htmlLength headheight=\the\headheight}\lthtmltypeout{}%
\lthtmltypeout{latex2htmlLength headsep=\the\headsep}\lthtmltypeout{}%
\lthtmltypeout{latex2htmlLength parskip=\the\parskip}\lthtmltypeout{}%
\lthtmltypeout{latex2htmlLength oddsidemargin=\the\oddsidemargin}\lthtmltypeout{}%
\makeatletter
\if@twoside\lthtmltypeout{latex2htmlLength evensidemargin=\the\evensidemargin}%
\else\lthtmltypeout{latex2htmlLength evensidemargin=\the\oddsidemargin}\fi%
\lthtmltypeout{}%
\makeatother
\setcounter{page}{1}
\onecolumn

% !!! IMAGES START HERE !!!

\stepcounter{section}
\stepcounter{section}
{\newpage\clearpage
\lthtmlfigureA{framed38}%
\begin{framed}
% latex2html id marker 38

    \centering
    \includegraphics[scale=0.6]{01.arrows.eps}
    \caption{Toy model for the need of human coordination. We have 
depicted a chamber separated in two parts by a moving piston. On the 
left there are many disordered arrows pointing at random directions. The 
arrows represent forces, so for total randomness the total 
force becomes zero. The temperature on the left is high, which means a great reservoir of
energy but in an unusable form. On the right we have represented four single forces, 
the main powers now in charge of the world, in a perfectly coordinated push 
to have most of the chamber (the world or its wealth) available for them. The temperature
on the right is clearly very low, without any hint of randomness. On the left 
part of the chamber there is still some space left because the 
disorganized forces produce some pressure, which is transmitted to the piston, as molecules in 
a gas do. If the right-arrows push too hard, the temperature on the left 
part of the chamber increases even more and so does its pressure, until some 
suffocating equilibrium is achieved. If the temperature of the left arrows increases 
for whatever reasons, as in a violent revolution, its pressure over the right-arrows increases as well, 
and humanity can gain some part of the chamber, but not much. What 
humanity, and the whole ecosystem needs is a de-thermalization of the 
left-arrows, so they can become perfectly organized, pushing to the right. You 
are probably one of these left-arrows right now, so why not aligning with 
the rest and start doing the {\it right} thing? If we achieve this 
state, the whole chamber will be available to all, and the current 
powers will be modified in such a way that they will work for the whole, 
not for a separate space of their own. So, instead of growing hot, in a revolutionary way, humanity needs a 
rational cooling until it freezes to a non-random situation. A smooth renaissance with critical thinking as
our main tool. We need to 
stop our lives for a moment, think with cold minds, and realize that hot responses are not 
efficient. Of course, not responding is not an option either! Only a pacific and illustrated transition from disorder to order is able to overcome 
the problems that otherwise will strangle us. But how is such a difficult and necessary transition achieved?}
 \end{framed}%
\lthtmlfigureZ
\lthtmlcheckvsize\clearpage}

\stepcounter{section}
{\newpage\clearpage
\lthtmlinlinemathA{tex2html_wrap_inline379}%
$7\cdot 10^9$%
\lthtmlinlinemathZ
\lthtmlcheckvsize\clearpage}

{\newpage\clearpage
\lthtmlinlinemathA{tex2html_wrap_inline381}%
$10^{13}$%
\lthtmlinlinemathZ
\lthtmlcheckvsize\clearpage}

{\newpage\clearpage
\lthtmlfigureA{framed62}%
\begin{framed}
% latex2html id marker 62

    \centering
    \includegraphics[scale=0.7]{02.trees.eps}
    \caption{Simplified schematics of branching examples. From left to 
right and up to bottom: 1.Trees develop branches and leaves exhibit 
branched nerve systems. Roots also are branched. In general, plants 
usually grow and develop using fractal-like branching systems. 2.Rivers 
converge until huge quantities of water are collected and delivered back to the sea. 
3.Air in lungs goes through a sophisticated branched system to bring 
oxygen molecules and expel carbon dioxide to and from every single red 
blood cell. 4.Fraudulent pyramid schemes follow branching structures as 
well, but don't be afraid because r0oT is not one of these frauds. You can join and check this by yourself. 
5.Blood reaches each cell of our bodies and is collected from each of them 
thanks to an extremely branched system of arteries and veins: the 
circulatory system. 6.Snowflakes are beautiful examples of branching and 
symmetry. A good metaphor of what we need today: a coldly planned growth following a beautiful pattern to bring the ice poles back on Earth. 
The question is, how can we exactly grow such an ambitious structure?}
\end{framed}%
\lthtmlfigureZ
\lthtmlcheckvsize\clearpage}

\stepcounter{section}
{\newpage\clearpage
\lthtmlfigureA{framed72}%
\begin{framed}
% latex2html id marker 72

    \centering
    \includegraphics[scale=0.17]{03.growth.eps}
    \caption{Growth of the first 5 layers of the r0oT network. Circles 
are nodes, occupied by a single person. Lines are connections, who will 
be performed through emails or other means of communication. In gray we 
mark some stations, the units of discussion and action. From left to 
right and top to bottom: 1.Layer 1 is formed by a single person, the 
node 1.1. Notice the nomenclature: the first digit indicates layer, 
while the second is for the membership within the layer. 2.The second 
layer involves two people, the nodes 2.1 and 2.2. Notice that the three 
depicted members form the first station, s1.1. The nomenclature for 
stations is as follows: first begin with `s', which denotes station, and 
then the name or address of the station agent. 3.In the third layer we 
add 6 more people, 3 of them to one branch and the other 3 to another 
branch. 4.The fourth layer involves 24 new people, spread in 6 
stations. 5.The fifth people needs 120 new people to be spread 
in 24 stations. Notice how rapidly the population grows in each layer. 
Pictures beyond layer 6 are almost unrenderable.
}
\end{framed}%
\lthtmlfigureZ
\lthtmlcheckvsize\clearpage}

\stepcounter{section}
{\newpage\clearpage
\lthtmlfigureA{framed95}%
\begin{framed}
% latex2html id marker 95

    \centering
    \includegraphics[scale=0.5]{04.informaction.eps}
    \caption{Scheme representing the flows of informaction through the human mind of a stick figure. The
sources of manipulated information arrive to the mind. If these memes do not encounter 
Critical Thinking (CT) resistance, they develop into actions that mostly benefit the meme's authors, not the subject acting. In the same way as a small bird that feeds a relatively giant cuckoo's baby even if
it is blatantly clear that it is not its child \cite{cuckoo}. In many ways we are even more blatantly
manipulated that those birds. However, if these memes encounter CT resistance, they
will not easily progress into actions that favor them, so they will neither reproduce nor
benefit their remote authors. Here, red means manipulation flow, while green is used for
freedom (free from manipulation). Depending on how free is our output we will feed
a red feedback mechanism, benefiting the current powers, or a green feedback
mechanism, in which other powers, better for Gaia and the people, can appear. Many
people think that people are evil, but this is just a very red meme! The truth
is that people are neither evil nor good in an intrinsic way, but strongly
manipulable, ductile and flexible. We need to help each other to bend our wrongly supposed
evil into good actual will. But how can informaction flow following the green feedback loop in order to raise our CT
and, as a consequence, our own freedom?}
\end{framed}%
\lthtmlfigureZ
\lthtmlcheckvsize\clearpage}

\stepcounter{section}
{\newpage\clearpage
\lthtmlfigureA{framed118}%
\begin{framed}
% latex2html id marker 118

    \centering
    \includegraphics[scale=0.3]{05.speed.eps}
    \caption{Infographic of some details of the r0oT network. The three left color circles represent
the angular and radial flows that informaction acquires. The black circles on the left show what
we mean by angular and radial directions. In the middle column we find the
numbers of how many people can be in each layer, as well as the percentage of the current total
population of about 7.4 billion people. Notice how, reaching layer 11 we could acquire a high
degree of coordination while involving about half of 1\%. We have also depicted two radial
arrows representing the radial gradient of abstraction/specificity that can naturally appear in the
network. Inner layers will probably tend to deal with more global topics while outer
layers will probably deal with more local, direct, practical subjects. However,
the structure of r0oT is thought in such a way that inner layers also act as propagators
of local topics while local layers act as promoters of global issues. 
There is also a radial arrow depicting locality. This means that the network has global
inner layers, with their members as mixed as possible throughout the world, while
external layers will engineered to be local, where network neighbors will be also real world
neighbors. Four colored radial arrows represent the 4 speeds and processing depths
of the reports: green for slow speed and high processing, yellow for medium speed
and medium processing, orange for high speed and low processing and red for maximum
speed and minimum processing. Every choice has a price. Now we move to what most people, in
layer 13, will experience by participating in r0oT.
}
\end{framed}%
\lthtmlfigureZ
\lthtmlcheckvsize\clearpage}

\stepcounter{section}
\stepcounter{section}
{\newpage\clearpage
\lthtmlfigureA{framed127}%
\begin{framed}
% latex2html id marker 127

    \centering
    \includegraphics[scale=0.55]{06.democracy.eps}
    \caption{Depiction of local vs global democracy. What affects a single 
station is voted in the station. What affects a branch must be voted by the 
whole branch. What affects the whole network must be voted by everyone. On the
right we have depicted some branches with dashed red lines. Branches always stem
from a specific station. Local
democracy allows the system to be very efficient while at the same time
to be democratic, but global democracy ensures that the same principles are
applied everywhere. Only in a world where the rights and obligations are
the same to everyone we can speak of true justice and democracy. In our
current world, there are many different countries with many different sets
of rights and laws, and it is not just that some are more fortunate than
others by the mere accident of being born here or there: it is that in the
countries with more `freedom' and `rights', we take advantage of the other
`lesser' countries to have cheap workers under unfair conditions so we can
obtain cheap and nice goods to buy. In this context, Gaia, the Earth, is the
last country on the list, the country that nobody takes care of, the one
that offers the cheapest products. We need to revert this order and put
Gaia first.    
}
\end{framed}%
\lthtmlfigureZ
\lthtmlcheckvsize\clearpage}

{\newpage\clearpage
\lthtmlfigureA{framed134}%
\begin{framed}
% latex2html id marker 134

    \centering
    \includegraphics[scale=0.45]{07.mobility.eps}
    \caption{Radial and angular mobility within the network. Radial permutations are
performed by local democracy while angular permutations are performed upon mutual 
agreement of the two willing members.}
\end{framed}%
\lthtmlfigureZ
\lthtmlcheckvsize\clearpage}

\stepcounter{section}
{\newpage\clearpage
\lthtmlfigureA{framed148}%
\begin{framed}
% latex2html id marker 148

    \centering
    \includegraphics[scale=0.8]{08.immunity_growth.eps}
    \caption{Depiction of the first four layers of the i1 network. On the left we show
the nodes (hexagons to distinguish from the circles of the main network) and their
connections. We also see that the nomenclature is the same as in the main network except
for the fact that we add the label i1 as the first 2 digits. On the left we can see the
stations that i1 members form, applying the same principles as in the main system.
}
\end{framed}%
\lthtmlfigureZ
\lthtmlcheckvsize\clearpage}

{\newpage\clearpage
\lthtmlfigureA{framed155}%
\begin{framed}
% latex2html id marker 155

    \centering
    \includegraphics[scale=0.7]{09.immunity.eps}
    \caption{Coupling of the main and the i1 networks. The main network is depicted
in black with circular nodes while the i1 network is depicted in green with hexagonal
nodes. The stations painted here are those of the main network, remarking how in each station there
is always present an immunity member of i1. We can see a side note pointing to the
possibility of building further immunity networks, i2, i3 and so on, in order to 
reinforce the robustness of the whole system. For now we will try only with i1, but
if necessary we can build the others. If they were to be created, they would have
again the same structure as the other two, and the coupling would be very simple: the
i(n+1) network would couple to the i(n) network as the i1 would couple to the main system. 
}
\end{framed}%
\lthtmlfigureZ
\lthtmlcheckvsize\clearpage}

\stepcounter{section}
{\newpage\clearpage
\lthtmlfigureA{framed172}%
\begin{framed}
% latex2html id marker 172

    \centering
    \includegraphics[scale=0.7]{10.specialists.eps}
    \caption{Specialist networks. We show 7 different specialist networks. We don't
show nomenclature even if it is simple enough: add a `sp' as a prefix and follow the
usual nomenclature. Immunity-specialist networks are only shown in the orange network,
and we don't show nomenclature either, which would have
the prefix spi1. As a suffix, all of them will have the name of the specialty. For
example, we may have sp.2.3.farming, which is the 2.3 node of the farming network.
Or sps.3.2.biology, which is the 3.2 station of the biology network. And so on. Notice
how there is a specialist at each station at a given visited branch. And notice as well
how the visits jump from one branch to the next in a counterclockwise angular fashion.
Depending on how many members the network has, it has a greater or lower degree, which
just indicates the main layer at which it attaches.  
}
\end{framed}%
\lthtmlfigureZ
\lthtmlcheckvsize\clearpage}

\stepcounter{section}

\end{document}
